\documentclass[11pt]{article}

\usepackage[a4paper]{geometry}
\geometry{left=2.0cm,right=2.0cm,top=2.5cm,bottom=2.5cm}

\usepackage{ctex} % 支持中文的LaTeX宏包
\usepackage{amsmath,amsfonts,graphicx,amssymb,bm,amsthm,mathrsfs,mathtools,breqn} % 数学公式和符号的宏包集合
\usepackage{algorithm,algorithmicx} % 算法和伪代码
\usepackage[noend]{algpseudocode} % 算法和伪代码
\usepackage{fancyhdr} % 自定义页眉页脚
\usepackage[framemethod=TikZ]{mdframed} % 创建带边框的框架
\usepackage{fontspec} % 字体设置
\usepackage{adjustbox} % 调整盒子大小
\usepackage{fontsize} % 设置字体大小
\usepackage{tikz,xcolor} % 绘制图形和使用颜色
\usepackage{multicol} % 多栏排版
\usepackage{multirow} % 表格中合并单元格
\usepackage{pdfpages} % 插入PDF文件
\usepackage{listings} % 在文档中插入源代码
\usepackage{wrapfig} % 文字绕排图片
\usepackage{bigstrut,multirow,rotating} % 支持在表格中使用特殊命令
\usepackage{booktabs} % 创建美观的表格
\usepackage{circuitikz} % 绘制电路图
\usepackage{zhnumber} % 中文序号(用于标题)
\usepackage{tabularx} % 表格折行
\usepackage{float} % 图片插入
\usepackage{longtable} % 长表格

\definecolor{dkgreen}{rgb}{0,0.6,0}
\definecolor{gray}{rgb}{0.5,0.5,0.5}
\definecolor{mauve}{rgb}{0.58,0,0.82}
\lstset{
  frame=tb,
  aboveskip=3mm,
  belowskip=3mm,
  showstringspaces=false,
  columns=fixed,
  framerule=1pt,
  rulecolor=\color{gray!35},
  backgroundcolor=\color{gray!5},
  basicstyle={\small\ttfamily},
  numbers=none,
  numberstyle=\tiny\color{gray},
  keywordstyle=\color{blue},
  commentstyle=\color{dkgreen},
  stringstyle=\color{mauve},
  breaklines=true,
  breakatwhitespace=true,
  tabsize=3,
}

% 轻松引用, 可以用\cref{}指令直接引用, 自动加前缀. 
% 例: 图片label为fig:1
% \cref{fig:1} => Figure.1
% \ref{fig:1}  => 1
\usepackage[capitalize]{cleveref}
% \crefname{section}{Sec.}{Secs.}
\Crefname{section}{Section}{Sections}
\Crefname{table}{Table}{Tables}
\crefname{table}{Table.}{Tabs.}

% \setmainfont{Palatino Linotype.ttf}
% \setCJKmainfont{SimHei.ttf}
% \setCJKsansfont{Songti.ttf}
% \setCJKmonofont{SimSun.ttf}
\punctstyle{kaiming}
% 偏好的几个字体, 可以根据需要自行加入字体ttf文件并调用

\renewcommand{\emph}[1]{\begin{kaishu}#1\end{kaishu}}

% 对 section 等环境的序号使用中文
\renewcommand \thesection{\zhnum{section}、}
\renewcommand \thesubsection{\arabic{subsection}}


%%%%%%%%%%%%%%%%%%%%%%%%%%%
%改这里可以修改实验报告表头的信息
\newcommand{\name}{陈琛、王荦璠、张钧玮}
\newcommand{\studentNum}{80}
\newcommand{\major}{计算机科学与技术}
\newcommand{\labNum}{4}
\newcommand{\labName}{异常与中断设计专题}
%%%%%%%%%%%%%%%%%%%%%%%%%%%

\begin{document}
\begin{center}
  \LARGE \bf 中国科学院大学 \\《计算机体系结构(研讨课)》实验报告
\end{center}

\begin{center}
  \emph{姓名} \underline{\makebox[12em][c]{\name}} 
  \emph{箱子号} \underline{\makebox[7em][c]{\studentNum}}
  \emph{专业} \underline{\makebox[15em][c]{\major}}\\
  \emph{实验项目编号} \underline{\makebox[3em][c]{\labNum}}
  \emph{实验名称} \underline{\makebox[30em][c]{\labName}}\\
\end{center}

\section{实验任务}

本次实验的目标是在已有的流水线CPU基础上,添加特权态指令中的异常和中断功能。具体需要实现:

\begin{itemize}
\item CSR模块,包括CRMD、PRMD、ESTAT、ERA、EENTRY、SAVE0~3、ECFG、BADV、TID、TCFG、TVAL和TICLR共15个状态寄存器;
\item 实现syscall和ertn指令,支持系统调用异常;
\item 实现地址错(ADEF)、地址非对齐(ALE)、断点(BRK)和指令不存在(INE)异常,以及外部线中断;
\item 实现csrrd、csrwr、csrxchg、rdcntvl.w、rdcntvh.w和rdcntid指令以访问状态寄存器。
\end{itemize}

\section{设计详细分析}

\subsection{总体设计思路}

在本次实验中,我们采用了分层次的设计方法,将特权态指令的实现分为三个关键部分。首先,新增独立的CSR模块,实现15个状态寄存器的硬件存储与访问逻辑。为保证精确异常,所有CSR写操作被推迟到WB级统一处理,并通过阻塞机制解决数据冲突。其次,在异常处理层面,需要扩展流水线控制逻辑以支持四种异常类型(ADEF/ALE/BRK/INE)和外部中断,通过ESTAT寄存器记录异常原因,由BADV寄存器保存错误地址。最后,在指令层面,需要修改译码模块,添加syscall、ertn和CSR访问指令(csrrd/csrwr/csrxchg/rdcntvl.w/rdcntvh.w/rdcntid)。

\subsection{处理器结构设计框图}

\begin{figure}[H]
  \centering
  \includegraphics[width=\textwidth]{fig/CPU_arch_prj4.jpg}
  \caption{处理器结构设计框图}
  \label{fig:processor_structure}
\end{figure}

\subsection{关键设计模块}

\subsubsection{CSR模块实现}

本次实验中按照讲义推荐的方式实现CSR模块,即将所有CSR集中到同一个模块中实现,除了提供统一的读写端口外,独立定义与硬件电路交互的信号接口。根据以上设计理念,CSR模块的接口定义如下:

\begin{lstlisting}[language=Verilog]
module CSR
  (
   input wire         clk,
   input wire         resetn,

   // 统一的CSR读写端口
   input wire         csr_re,
   input wire [13:0]  csr_num,
   output wire [31:0] csr_rvalue,
   input wire         csr_we,
   input wire [31:0]  csr_wmask,
   input wire [31:0]  csr_wvalue,

   // 硬件交互端口
   input wire [7:0]   hw_int_in,   // 硬中断输入
   input wire         ipi_int_in,  // 核间中断输入
   input wire [31:0]  coreid_in,   // 核心编号
   output wire [31:0] ex_entry,    // 送往IF级的异常处理入口
   output wire [31:0] era,         // 送往IF级的异常返回地址
   output wire        has_int,     // 送往ID级的中断有效信号
   // 来自写回级的异常处理信号
   input wire         ertn_flush,
   input wire         wb_ex,
   input wire [31:0]  wb_pc,
   input wire [31:0]  wb_vaddr,
   input wire [5:0]   wb_ecode,
   input wire [8:0]   wb_esubcode
   );
\end{lstlisting}

接下来,我们以CRMD寄存器为例,说明CSR的基本实现方式。对于CRMD寄存器而言,PLV域和IE域可以被软件指令更新,同时也会在异常产生和返回时更新;DA、PG、DATF、DATM域暂时实现为固定值。具体实现的Verilog代码如下:

\begin{lstlisting}[language=Verilog]
// CRMD
reg [1:0] csr_crmd_plv;
reg       csr_crmd_ie;
always @(posedge clk) begin
    if (~resetn) begin
        csr_crmd_plv <= 2'b0;
        csr_crmd_ie  <= 1'b0;
    end else if (wb_ex) begin
        // 异常发生
        csr_crmd_plv <= 2'b0;
        csr_crmd_ie  <= 1'b0;
    end else if (ertn_flush) begin
        // 异常返回
        csr_crmd_plv <= csr_prmd_pplv;
        csr_crmd_ie  <= csr_prmd_pie;
    end else if (csr_we && csr_num == `CSR_CRMD) begin
        // 软件修改
        csr_crmd_plv <= ( csr_wmask[`CSR_CRMD_PLV] & csr_wvalue[`CSR_CRMD_PLV]) |
                        (~csr_wmask[`CSR_CRMD_PLV] & csr_crmd_plv);
        csr_crmd_ie  <= ( csr_wmask[`CSR_CRMD_IE]  & csr_wvalue[`CSR_CRMD_IE] ) |
                        (~csr_wmask[`CSR_CRMD_IE]  & csr_crmd_ie );
    end
end

wire       csr_crmd_da   = 1'b1;
wire       csr_crmd_pg   = 1'b0;
wire [1:0] csr_crmd_datf = 2'b0;
wire [1:0] csr_crmd_datm = 2'b0;

wire [31:0] csr_crmd =
            // 31:9           8:7            6:5            4            3            2           1:0
            {23'd0, csr_crmd_datm, csr_crmd_datf, csr_crmd_pg, csr_crmd_da, csr_crmd_ie, csr_crmd_plv};
\end{lstlisting}

其他CSR的实现与CRMD大同小异,此处不再赘述。

\subsubsection{异常与中断的流水线实现}

为了支持精确异常,我们采用了分布式异常检测和集中式异常处理的设计思路。异常检测逻辑分布在各流水级中,异常信息随着指令在流水线中传递,最终在写回级统一处理。

\paragraph{异常信息的流水线传递}

我们在流水线各级之间扩展了异常传递字段,包含异常有效位、异常编码、异常子编码和异常返回标志。在macros.h中定义如下:

\begin{lstlisting}[language=Verilog]
// Exception/ERTN pipeline fields: {ex_valid, ecode[5:0], esubcode[8:0], is_ertn}
`define EX_FIELDS_LEN 17

// Exception codes
`define ECODE_INT  6'd0    // Interrupt
`define ECODE_ADE  6'd8    // Address Error
`define ECODE_ALE  6'd9    // Address Alignment Error  
`define ECODE_SYS  6'd11   // System call
`define ECODE_BRK  6'd12   // Breakpoint
`define ECODE_INE  6'd13   // Instruction Not Exist

`define ESUBCODE_ADEF  9'd0  // Address error for instruction fetch
\end{lstlisting}

这些异常字段被添加到各级流水线寄存器中,从IF级开始逐级传递至WB级,确保异常信息与触发异常的指令保持同步。

\paragraph{取指地址错异常(ADEF)检测}

取指地址错异常在IFU模块中检测。根据LoongArch规范,指令PC必须4字节对齐。我们在取指级检测PC的最低两位,若不为00则产生ADEF异常:

\begin{lstlisting}[language=Verilog]
// Exception detection: ADEF - Address error for instruction fetch
// PC must be word-aligned (lowest 2 bits must be 00)
assign if_ex_adef = (if_pc[1:0] != 2'b00);
assign if_ex_valid = if_ex_adef;
assign if_ecode = if_ex_adef ? `ECODE_ADE : 6'd0;
assign if_esubcode = if_ex_adef ? `ESUBCODE_ADEF : 9'd0;

// Pack exception fields into pipeline register
wire [`EX_FIELDS_LEN-1:0] if_ex_fields = 
    {if_ex_valid, if_ecode, if_esubcode, if_is_ertn};
\end{lstlisting}

该异常信息随取指的指令和PC一起打包,通过\texttt{if\_to\_id\_zip}传递到译码级。

\paragraph{断点异常(BRK)、指令不存在异常(INE)和中断的检测}

这三类异常在IDU模块的译码级进行检测。对于断点异常,我们通过指令译码识别\texttt{break}指令;对于指令不存在异常,我们维护了一个\texttt{inst\_known}信号,通过对所有已实现指令的OR运算来判断当前指令是否已实现;对于中断,我们从CSR模块接收\texttt{has\_int}信号并在ID级标记到指令上。

\begin{lstlisting}[language=Verilog]
// Instruction decode
wire inst_break = op_31_26_d[0] & op_25_22_d[0] & op_21_20_d[2] & op_19_15_d[20];

// Check all implemented instructions
wire inst_known = inst_add_w | inst_sub_w | ... | inst_rdcntvh;

// Exception and interrupt generation
wire id_ex_int     = has_int & id_valid;         // Interrupt exception
wire id_ex_syscall = inst_syscall;               // System call exception
wire id_ex_break   = inst_break;                 // Breakpoint exception
wire id_ex_ine     = ~inst_known & id_valid;     // Instruction not exist exception

// Priority encoding for exception codes
wire id_ex_valid = if_ex_valid | id_ex_int | id_ex_syscall | 
                   id_ex_break | id_ex_ine;
wire [5:0] id_ecode = if_ex_valid   ? if_ecode :
                      id_ex_int     ? `ECODE_INT :
                      id_ex_syscall ? `ECODE_SYS :
                      id_ex_break   ? `ECODE_BRK :
                      id_ex_ine     ? `ECODE_INE : 6'd0;
\end{lstlisting}

这里采用了优先级编码,当多个异常同时发生时,程序顺序靠前的异常(来自IF级的异常)优先级最高。此外,为防止异常指令修改处理器状态,我们在\texttt{gr\_we}信号中加入了\texttt{\textasciitilde{}id\_ex\_valid}条件,确保发生异常时不写入通用寄存器。

\paragraph{地址非对齐异常(ALE)检测}

地址非对齐异常在EXEU模块的执行级检测。该异常针对load/store指令,要求半字访问地址2字节对齐,字访问地址4字节对齐:

\begin{lstlisting}[language=Verilog]
// Address alignment check for memory operations
wire is_mem_op = (exe_mem_op != 4'd0);
wire is_half_op = (exe_mem_op == 4'd1) | (exe_mem_op == 4'd5) | 
                  (exe_mem_op == 4'd9);  // ld.h, st.h, ld.hu
wire is_word_op = (exe_mem_op == 4'd2) | (exe_mem_op == 4'd6); // ld.w, st.w

wire addr_align_error = is_mem_op & (
    (is_half_op & exe_alu_result[0]) |           // Half-word: 2-byte aligned
    (is_word_op & (|exe_alu_result[1:0]))        // Word: 4-byte aligned
);
wire exe_gen_ex_valid = addr_align_error;
wire [5:0] exe_gen_ecode = addr_align_error ? `ECODE_ALE : 6'd0;

// Priority: local exception > pipeline exception
wire exe_to_mem_ex_valid = exe_gen_ex_valid ? 1'b1 : exe_ex_valid;
wire [5:0] exe_to_mem_ecode = exe_gen_ex_valid ? exe_gen_ecode : exe_ecode;
\end{lstlisting}

检测到地址非对齐时,本级产生的异常优先于从上一级传递来的异常。同时,访存地址(\texttt{exe\_alu\_result})被添加到流水线寄存器中作为\texttt{vaddr}字段,用于更新BADV寄存器。

\paragraph{SYSCALL和ERTN指令的实现}

\texttt{syscall}指令用于触发系统调用异常,\texttt{ertn}指令用于从异常处理返回。这两条指令在IDU中译码后,分别标记为系统调用异常和异常返回标志,随流水线传递至WB级处理:

\begin{lstlisting}[language=Verilog]
// Instruction decode
wire inst_syscall = op_31_26_d[0] & op_25_22_d[0] & op_21_20_d[2] & 
                    op_19_15_d[22];
wire inst_ertn = op_31_26_d[1] & op_25_22_d[9] & op_21_20_d[0] & 
                 op_19_15_d[16] & (rk == 5'h0e) & (~|rj) & (~|rd);

// Mark exception/ertn for pipeline propagation
wire id_ex_syscall = inst_syscall;
wire id_is_ertn = if_is_ertn | inst_ertn;
\end{lstlisting}

在WB级,当\texttt{syscall}到达时产生系统调用异常,CSR模块更新相关状态寄存器(CRMD、PRMD、ESTAT、ERA等)并生成异常入口地址;当\texttt{ertn}到达时,CSR模块恢复CRMD寄存器并生成返回地址。这两种情况都会产生全局冲刷信号\texttt{flush},清空流水线并跳转到目标地址。

\paragraph{写回级的异常统一处理}

所有异常最终在WBU模块中统一提交。WB级将异常信息传递给CSR模块,并产生全局冲刷信号:

\begin{lstlisting}[language=Verilog]
// Exception/ERTN outputs
assign wb_ex_valid = wb_valid & (wb_ex_valid_r | wb_gen_ex_valid);
assign wb_ex_pc = wb_pc;
assign wb_vaddr = wb_vaddr_r;  // For BADV register
assign wb_is_ertn = wb_valid & wb_is_ertn_r;
\end{lstlisting}

这些信号连接到CSR模块,触发异常处理流程。同时,\texttt{wb\_ex\_valid}或\texttt{wb\_is\_ertn}信号会生成全局冲刷,确保异常指令之后的所有指令被取消执行,实现精确异常。

\subsubsection{CSR读写指令的实现}

\paragraph{CSR读写信息的流水线传递}

为方便起见,当前阶段统一在WB阶段读写CSR寄存器。CSR读写指令在ID阶段被解码,产生如下的流水线信号并随流水线传递至WB阶段,由WB统一执行读写。

\begin{lstlisting}[language=Verilog]
// CSR pipleline fields: {csr_read, csr_we, csr_num[13:0], csr_wmask[31:0], csr_wvalue[31:0]}
`define CSR_FIELDS_LEN 80
\end{lstlisting}

\paragraph{阻塞解决CSR指令引发的数据相关}

CSR指令有可能引发数据相关。比如前一条指令将CSR中的值保存进寄存器1,而后一条指令以寄存器1为源操作数,在后一条指令进入ID阶段解码的时候,前一条指令还没有进入WB阶段读取CSR寄存器,由此触发数据相关。

由于CSR读写指令的密度稀少,针对此进行优化的收益低,故简单起见,本实验中采用阻塞解决。在ID阶段后的各个阶段前递给ID的数据中,设置 res\_from\_exe, res\_from\_mem, res\_from\_wb信号,之后ID阶段负责识别这些信号,使当前级陷入阻塞,从而阻塞整条流水线,避免数据相关。ID阶段的代码如下:

\begin{lstlisting}[language=Verilog]
assign id_stall = ((exe_res_from_mem | exe_from_csr) & ((conflict_r1_exe & need_r1) | (conflict_r2_exe & need_r2)))
|
     ((mem_res_from_mem | mem_from_csr) & ((conflict_r1_mem & need_r1) | (conflict_r2_mem & need_r2))) |
      ((wb_from_csr)                     & ((conflict_r1_wb  & need_r1) | (conflict_r2_wb  & need_r2)));
\end{lstlisting}

\section{实验过程中遇到的问题}

\subsection{异常信号的前递问题}

在EXE级需要向内存发送读写请求,此时若在EXE级之后的指令若产生异常,则当前EXE级中的指令应该被冲刷掉,因此请求不应该被发出。由于在当前的CPU当中,在MEM与WB阶段不额外产生新的异常,故可以通过将MEM与WB阶段的异常信号前递,在EXE级别统一判断是否接发送请求。原设计中缺少这一部分,导致报错。修复后,MEM与WB级前递的信号为 mem\_ex与 wb\_ex,之后由如下逻辑判断是否可以正常发送请求:

\begin{lstlisting}[language=Verilog]
wire exe_multicycle_ok = exe_valid & ~(mem_ex | wb_ex | exe_to_mem_ex_valid | exe_to_mem_is_ertn);

assign data_sram_en = (exe_res_from_mem | exe_mem_op[2]) & exe_multicycle_ok;
\end{lstlisting}

\subsection{异常发生后的多周期指令处理问题}

在syscall进入WB级进行异常处理时,后续的2条指令可能分别都进入了EXE级和MEM级。特别地,如果这两条指令中存在多周期指令(如mul, div, 访存等),可能已经开始执行。此时如果需要冲刷,不能简单地拉低valid(否则多周期状态机可能已经开始执行,状态机执行完成后输出错误结果)。因此,需要通过异常前递信号判断先前的指令是否已经产生了异常。若已经产生,则不执行后续的多周期指令。

\subsection{先前遗留的译码逻辑问题}

在原先的实现中,由于分支指令在ID阶段有单独的比较器,所以没有解码产生相应的alu\_op,导致基于alu\_op信号的need\_r1信号出现问题,进而导致ID阶段的stall逻辑在一些分支指令上没有正确触发,如下图pc=1c00800c所示是beq指令,它的need\_r1信号为0

\begin{figure}[H]
  \centering
  \includegraphics[width=\textwidth]{fig/decode_logic_problem_demo.jpg}
  \caption{译码逻辑问题示例图}
  \label{fig:processor_structure}
\end{figure}

在ID阶段完善 alu\_op 信号的译码,即可修复问题:

\begin{lstlisting}[language=Verilog]
    // ALU operation encoding (extended to 18 bits for multiply/divide)
    assign alu_op[0]  = inst_add_w | inst_addi_w | ty_M |
                        inst_jirl  | inst_bl     | inst_pcaddu12i;
    assign alu_op[1]  = inst_sub_w | inst_bne    | inst_beq;
    assign alu_op[2]  = inst_slt   | inst_slti   | inst_blt    | inst_bge;
    assign alu_op[3]  = inst_sltu  | inst_sltui  | inst_bltu   | inst_bgeu;
    assign alu_op[4]  = inst_and   | inst_andi;
    assign alu_op[5]  = inst_nor;
    assign alu_op[6]  = inst_or    | inst_ori;
    assign alu_op[7]  = inst_xor   | inst_xori;
    assign alu_op[8]  = inst_sll_w | inst_slli_w;
    assign alu_op[9]  = inst_srl_w | inst_srli_w;
    assign alu_op[10] = inst_sra_w | inst_srai_w;
    assign alu_op[11] = inst_lu12i_w;
\end{lstlisting}


\subsection{ESTAT寄存器高阻态问题}

在编写完实验13的代码,运行仿真测试时,发现ESTAT信号初始化为高阻态:

\begin{figure}[H]
  \centering
  \includegraphics[width=\textwidth]{fig/estat_problem_demo.jpg}
  \caption{ESTAT初始化问题示例图}
  \label{fig:processor_structure}
\end{figure}

观察发现其中第12与2~9位为高阻态,对应hw\_init\_in与ipi\_init\_in。原因是这两个信号一路被拉到了顶层封装my\_cputop的接口,超出了最高层实例化的范围。将其在my\_cputop中手动设置为0即可。

\section{实验结果}

所有测试均通过验证,证明了我们实现的CPU能够正确支持异常和中断处理。

\section{小组成员分工}

陈琛:实现了CSR模块。修复EXE多周期指令问题。

张钧玮:负责CSR指令与计时器指令的实现,添加异常信号前递机制。

王荦璠:负责异常和中断的流水线实现。实现了syscall和ertn指令,设计了异常信息在流水线中的传递机制。

\end{document}
