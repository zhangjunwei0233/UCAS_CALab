\documentclass[11pt]{article}

\usepackage[a4paper]{geometry}
\geometry{left=2.0cm,right=2.0cm,top=2.5cm,bottom=2.5cm}

\usepackage{ctex} % 支持中文的LaTeX宏包
\usepackage{amsmath,amsfonts,graphicx,amssymb,bm,amsthm,mathrsfs,mathtools,breqn} % 数学公式和符号的宏包集合
\usepackage{algorithm,algorithmicx} % 算法和伪代码
\usepackage[noend]{algpseudocode} % 算法和伪代码
\usepackage{fancyhdr} % 自定义页眉页脚
\usepackage[framemethod=TikZ]{mdframed} % 创建带边框的框架
\usepackage{fontspec} % 字体设置
\usepackage{adjustbox} % 调整盒子大小
\usepackage{fontsize} % 设置字体大小
\usepackage{tikz,xcolor} % 绘制图形和使用颜色
\usepackage{multicol} % 多栏排版
\usepackage{multirow} % 表格中合并单元格
\usepackage{pdfpages} % 插入PDF文件
\usepackage{listings} % 在文档中插入源代码
\usepackage{wrapfig} % 文字绕排图片
\usepackage{bigstrut,multirow,rotating} % 支持在表格中使用特殊命令
\usepackage{booktabs} % 创建美观的表格
\usepackage{circuitikz} % 绘制电路图
\usepackage{zhnumber} % 中文序号(用于标题)
\usepackage{tabularx} % 表格折行
\usepackage{float} % 图片插入
\usepackage{longtable} % 长表格

\definecolor{dkgreen}{rgb}{0,0.6,0}
\definecolor{gray}{rgb}{0.5,0.5,0.5}
\definecolor{mauve}{rgb}{0.58,0,0.82}
\lstset{
  frame=tb,
  aboveskip=3mm,
  belowskip=3mm,
  showstringspaces=false,
  columns=fixed,
  framerule=1pt,
  rulecolor=\color{gray!35},
  backgroundcolor=\color{gray!5},
  basicstyle={\small\ttfamily},
  numbers=none,
  numberstyle=\tiny\color{gray},
  keywordstyle=\color{blue},
  commentstyle=\color{dkgreen},
  stringstyle=\color{mauve},
  breaklines=true,
  breakatwhitespace=true,
  tabsize=3,
}

% 轻松引用, 可以用\cref{}指令直接引用, 自动加前缀. 
% 例: 图片label为fig:1
% \cref{fig:1} => Figure.1
% \ref{fig:1}  => 1
\usepackage[capitalize]{cleveref}
% \crefname{section}{Sec.}{Secs.}
\Crefname{section}{Section}{Sections}
\Crefname{table}{Table}{Tables}
\crefname{table}{Table.}{Tabs.}

% \setmainfont{Palatino Linotype.ttf}
% \setCJKmainfont{SimHei.ttf}
% \setCJKsansfont{Songti.ttf}
% \setCJKmonofont{SimSun.ttf}
\punctstyle{kaiming}
% 偏好的几个字体, 可以根据需要自行加入字体ttf文件并调用

\renewcommand{\emph}[1]{\begin{kaishu}#1\end{kaishu}}

% 对 section 等环境的序号使用中文
\renewcommand \thesection{\zhnum{section}、}
\renewcommand \thesubsection{\arabic{subsection}}


%%%%%%%%%%%%%%%%%%%%%%%%%%%
%改这里可以修改实验报告表头的信息
\newcommand{\name}{陈琛、王荦璠、张钧玮}
\newcommand{\studentNum}{80}
\newcommand{\major}{计算机科学与技术}
\newcommand{\labNum}{4}
\newcommand{\labName}{异常与中断设计专题}
%%%%%%%%%%%%%%%%%%%%%%%%%%%

\begin{document}
\begin{center}
  \LARGE \bf 中国科学院大学 \\《计算机体系结构(研讨课)》实验报告
\end{center}

\begin{center}
  \emph{姓名} \underline{\makebox[12em][c]{\name}} 
  \emph{箱子号} \underline{\makebox[7em][c]{\studentNum}}
  \emph{专业} \underline{\makebox[15em][c]{\major}}\\
  \emph{实验项目编号} \underline{\makebox[3em][c]{\labNum}}
  \emph{实验名称} \underline{\makebox[30em][c]{\labName}}\\
\end{center}

\section{实验任务}

本次实验的目标是在已有的流水线CPU基础上,添加特权态指令中的异常和中断功能。具体需要实现:

\begin{itemize}
\item CSR模块,包括CRMD、PRMD、ESTAT、ERA、EENTRY、SAVE0~3、ECFG、BADV、TID、TCFG、TVAL和TICLR共15个状态寄存器;
\item 实现syscall和ertn指令,支持系统调用异常;
\item 实现地址错(ADEF)、地址非对齐(ALE)、断点(BRK)和指令不存在(INE)异常,以及外部线中断;
\item 实现csrrd、csrwr、csrxchg、rdcntvl.w、rdcntvh.w和rdcntid指令以访问状态寄存器。
\end{itemize}

\section{设计详细分析}

\subsection{总体设计思路}

在本次实验中,我们采用了分层次的设计方法,将特权态指令的实现分为三个关键部分。首先,新增独立的CSR模块,实现15个状态寄存器的硬件存储与访问逻辑。为保证精确异常,所有CSR写操作被推迟到WB级统一处理,并通过阻塞机制解决数据冲突。其次,在异常处理层面,需要扩展流水线控制逻辑以支持四种异常类型(ADEF/ALE/BRK/INE)和外部中断,通过ESTAT寄存器记录异常原因,由BADV寄存器保存错误地址。最后,在指令层面,需要修改译码模块,添加syscall、ertn和CSR访问指令(csrrd/csrwr/csrxchg/rdcntvl.w/rdcntvh.w/rdcntid)。

\subsection{处理器结构设计框图}

% \begin{figure}[H]
%   \centering
%   \includegraphics[width=\textwidth]{fig/CPU_arch_prj4.png}
%   \caption{处理器结构设计框图}
%   \label{fig:processor_structure}
% \end{figure}

\subsection{关键设计模块}

\subsubsection{CSR模块实现}

本次实验中按照讲义推荐的方式实现CSR模块,即将所有CSR集中到同一个模块中实现,除了提供统一的读写端口外,独立定义与硬件电路交互的信号接口。根据以上设计理念,CSR模块的接口定义如下:

\begin{lstlisting}[language=Verilog]
module CSR
  (
   input wire         clk,
   input wire         resetn,

   // 统一的CSR读写端口
   input wire         csr_re,
   input wire [13:0]  csr_num,
   output wire [31:0] csr_rvalue,
   input wire         csr_we,
   input wire [31:0]  csr_wmask,
   input wire [31:0]  csr_wvalue,

   // 硬件交互端口
   input wire [7:0]   hw_int_in,   // 硬中断输入
   input wire         ipi_int_in,  // 核间中断输入
   input wire [31:0]  coreid_in,   // 核心编号
   output wire [31:0] ex_entry,    // 送往IF级的异常处理入口
   output wire [31:0] era,         // 送往IF级的异常返回地址
   output wire        has_int,     // 送往ID级的中断有效信号
   // 来自写回级的异常处理信号
   input wire         ertn_flush,
   input wire         wb_ex,
   input wire [31:0]  wb_pc,
   input wire [31:0]  wb_vaddr,
   input wire [5:0]   wb_ecode,
   input wire [8:0]   wb_esubcode
   );
\end{lstlisting}

接下来,我们以CRMD寄存器为例,说明CSR的基本实现方式。对于CRMD寄存器而言,PLV域和IE域可以被软件指令更新,同时也会在异常产生和返回时更新;DA、PG、DATF、DATM域暂时实现为固定值。具体实现的Verilog代码如下:

\begin{lstlisting}[language=Verilog]
// CRMD
reg [1:0] csr_crmd_plv;
reg       csr_crmd_ie;
always @(posedge clk) begin
    if (~resetn) begin
        csr_crmd_plv <= 2'b0;
        csr_crmd_ie  <= 1'b0;
    end else if (wb_ex) begin
        // 异常发生
        csr_crmd_plv <= 2'b0;
        csr_crmd_ie  <= 1'b0;
    end else if (ertn_flush) begin
        // 异常返回
        csr_crmd_plv <= csr_prmd_pplv;
        csr_crmd_ie  <= csr_prmd_pie;
    end else if (csr_we && csr_num == `CSR_CRMD) begin
        // 软件修改
        csr_crmd_plv <= ( csr_wmask[`CSR_CRMD_PLV] & csr_wvalue[`CSR_CRMD_PLV]) |
                        (~csr_wmask[`CSR_CRMD_PLV] & csr_crmd_plv);
        csr_crmd_ie  <= ( csr_wmask[`CSR_CRMD_IE]  & csr_wvalue[`CSR_CRMD_IE] ) |
                        (~csr_wmask[`CSR_CRMD_IE]  & csr_crmd_ie );
    end
end

wire       csr_crmd_da   = 1'b1;
wire       csr_crmd_pg   = 1'b0;
wire [1:0] csr_crmd_datf = 2'b0;
wire [1:0] csr_crmd_datm = 2'b0;

wire [31:0] csr_crmd =
            // 31:9           8:7            6:5            4            3            2           1:0
            {23'd0, csr_crmd_datm, csr_crmd_datf, csr_crmd_pg, csr_crmd_da, csr_crmd_ie, csr_crmd_plv};
\end{lstlisting}

其他CSR的实现与CRMD大同小异,此处不再赘述。

\subsubsection{其他设计模块}

\section{实验过程中遇到的问题}

\subsection{异常发生后的多周期指令处理问题}

在syscall进入WB级进行异常处理时,后续的2条指令可能分别都进入了EXE级和MEM级。特别地,如果这两条指令中存在多周期指令(如mul, div, 访存等),可能已经开始执行。此时如果需要冲刷,不能简单地拉低valid(否则多周期状态机可能已经开始执行,状态机执行完成后输出错误结果)。因此,需要通过异常前递信号判断先前的指令是否已经产生了异常。若已经产生,则不执行后续的多周期指令。

\section{实验结果}

所有测试均通过验证,证明了我们实现的CPU能够正确支持异常和中断处理。

\section{小组成员分工}

陈琛:实现了CSR模块。

张钧玮:

王荦璠:

\end{document}
